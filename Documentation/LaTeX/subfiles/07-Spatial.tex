\documentclass[../LogosReference.tex]{subfiles}
\begin{document}

% ============================================================================
% Section: Spatial Extension
% ============================================================================

\section{Spatial Extension}\label{sec:spatial}

\textsc{[Details pending development]}

The Spatial Extension extends the Core Extension with structures for spatial reasoning and location.

% ------------------------------------------------------------
% Frame Extension
% ------------------------------------------------------------

\subsection{Frame Extension}\label{sec:spatial-frame}

\textsc{[Details pending development]}

The spatial frame extends the core frame with:
\begin{itemize}
  \item \textbf{Location space} $L$ = set of spatial locations
  \item \textbf{Spatial relations}: adjacency, containment, distance
\end{itemize}

\begin{question}
What spatial primitives are required? Should locations be mereological (with parts) or set-theoretic?
\end{question}

% ------------------------------------------------------------
% Operators
% ------------------------------------------------------------

\subsection{Operators}\label{sec:spatial-operators}

\begin{table}[h]
\centering
\begin{tabular}{ll}
\toprule
\textbf{Operator} & \textbf{Intended Reading} \\
\midrule
$\textbf{Here}(\metaA)$ & $\metaA$ holds at the current location \\
$\textbf{Somewhere}(\metaA)$ & $\metaA$ holds at some location \\
$\textbf{Everywhere}(\metaA)$ & $\metaA$ holds at all locations \\
$\textbf{Near}(\metaA)$ & $\metaA$ holds at an adjacent location \\
\bottomrule
\end{tabular}
\caption{Spatial operators}
\label{tab:spatial-operators}
\end{table}

\textsc{[Full semantic clauses for spatial operators pending specification]}

\end{document}
