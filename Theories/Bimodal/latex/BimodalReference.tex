% ============================================================================
% BimodalReference.tex
% Bimodal TM Logic: A Reference Manual
%
% This document provides the formal specification of the Bimodal TM logic,
% a bimodal logic combining S5 metaphysical modality with linear temporal
% operators, as implemented in the Bimodal/ directory.
% ============================================================================

\documentclass[11pt]{article}

% ============================================================================
% Packages
% ============================================================================

% Mathematics
\usepackage{amsmath}
\usepackage{amsthm}
\usepackage{amssymb}

% Document structure
\usepackage{subfiles}

% Tables
\usepackage{booktabs}
\usepackage{array}

% Lists
\usepackage{enumitem}

% Custom packages (found via TEXINPUTS configured in latexmkrc)
\usepackage{bimodal-notation}
\usepackage{formatting}

% TikZ for diagrams (loaded by tcolorbox; add libraries here)
\usetikzlibrary{positioning,arrows.meta,shapes}

% References
\usepackage{cleveref}

% ============================================================================
% Theorem Environments
% ============================================================================

\theoremstyle{definition}
\newtheorem{definition}{Definition}[section]

\theoremstyle{plain}
\newtheorem{theorem}[definition]{Theorem}
\newtheorem{lemma}[definition]{Lemma}
\newtheorem{axiom}[definition]{Axiom}

\theoremstyle{remark}
\newtheorem{remark}[definition]{Remark}

% ============================================================================
% Document Info
% ============================================================================

% Custom horizontal rule command
\newcommand{\HRule}{\rule{\linewidth}{0.5mm}}

% ============================================================================
% Begin Document
% ============================================================================

\begin{document}

% Custom title page
\begin{titlepage}
\begin{center}

\vspace*{2cm}

\HRule\\[0.4cm]
{\Huge \bfseries Bimodal Reference Manual}\\[0.2cm]
\HRule\\[1cm]

{\LARGE\itshape A Logic for Tense and Modality}\\[1cm]


{\large\itshape Benjamin Brast-McKie}\\[0.15cm]
\texttt{\href{https://www.benbrastmckie.com}{www.benbrastmckie.com}}\\[0.15cm]
{--- \today\ ---}\\[1cm]

\vfill

{\normalsize\bfseries Primary Reference:}\\[0.3cm]
  \href{https://www.benbrastmckie.com/wp-content/uploads/2025/11/possible_worlds.pdf}{\textit{``The Construction of Possible Worlds''}}, Brast-McKie, (under review), 2025.\\[1cm]

\end{center}
\end{titlepage}
\thispagestyle{empty}

{

\begin{abstract}
\noindent
This reference manual provides the formal specification of the Bimodal logic \textbf{TM} for tense and modality as implemented in the \proofchecker{} project.
\textbf{TM} is a bimodal logic combining an S5 historical necessity operator with linear temporal operators for the past and future tenses.
Soundness and the deduction theorem are established.
Completeness is proven via the semantic canonical model approach: the Lindenbaum lemma, truth lemma, and weak completeness theorem are all proven.
The key result \texttt{semantic\_weak\_completeness} demonstrates that validity implies derivability.
\end{abstract}

\pagestyle{empty}
\tableofcontents
\cleardoublepage
}

% ============================================================================
% Content
% ============================================================================

\subfile{subfiles/00-Introduction}
\subfile{subfiles/01-Syntax}
\subfile{subfiles/02-Semantics}
\subfile{subfiles/03-ProofTheory}
\subfile{subfiles/04-Metalogic}
\subfile{subfiles/05-Theorems}
\subfile{subfiles/06-Notes}

\end{document}
