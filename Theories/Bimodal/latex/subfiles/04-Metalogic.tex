\documentclass[../BimodalReference.tex]{subfiles}
\begin{document}

\section{Metalogic}

\subsection{Soundness}

The soundness theorem establishes that derivability implies semantic validity.

\begin{theorem}[Soundness]
If $\context \proves \varphi$ then $\context \satisfies \varphi$.
\end{theorem}

The proof proceeds by induction on the derivation structure:
\begin{itemize}
  \item \textbf{Axioms}: Each of the 14 axiom schemata is proven valid
  \item \textbf{Assumptions}: Assumed formulas are true by hypothesis
  \item \textbf{Modus ponens}: Validity preserved under implication elimination
  \item \textbf{Necessitation}: Valid formulas become necessarily valid
  \item \textbf{Temporal necessitation}: Valid formulas become always-future valid
  \item \textbf{Temporal duality}: Past-future swap preserves validity
  \item \textbf{Weakening}: Adding premises preserves semantic consequence
\end{itemize}

\begin{center}
\begin{tabular}{@{}lll@{}}
\toprule
Axiom Validity & Lean Theorem & Technique \\
\midrule
\texttt{prop\_k\_valid} & Propositional K & Propositional reasoning \\
\texttt{prop\_s\_valid} & Propositional S & Propositional reasoning \\
\texttt{ex\_falso\_valid} & EFQ & Vacuous implication \\
\texttt{peirce\_valid} & Peirce & Classical case analysis \\
\texttt{modal\_t\_valid} & MT & Reflexivity of accessibility \\
\texttt{modal\_4\_valid} & M4 & Transitivity of accessibility \\
\texttt{modal\_b\_valid} & MB & Symmetry of accessibility \\
\texttt{modal\_5\_collapse\_valid} & M5 & S5 equivalence structure \\
\texttt{modal\_k\_dist\_valid} & MK & Distribution \\
\texttt{temp\_k\_dist\_valid} & TK & Temporal distribution \\
\texttt{temp\_4\_valid} & T4 & Transitivity of time \\
\texttt{temp\_a\_valid} & TA & Temporal connectedness \\
\texttt{temp\_l\_valid} & TL & Always implies recurrence \\
\texttt{modal\_future\_valid} & MF & Time-shift invariance \\
\texttt{temp\_future\_valid} & TF & Time-shift invariance \\
\bottomrule
\end{tabular}
\end{center}

The MF and TF axioms use time-shift invariance (via \texttt{WorldHistory.time\_shift}) to relate truth at different times.

\subsection{Deduction Theorem}

\begin{theorem}[Deduction Theorem]
If $A :: \context \proves B$ then $\context \proves A \imp B$.
\end{theorem}

The proof uses well-founded induction on derivation height, handling each rule:
\begin{itemize}
  \item \textbf{Axiom}: Use S axiom to weaken
  \item \textbf{Assumption}: Identity if same, S axiom if different
  \item \textbf{Modus ponens}: Use K axiom distribution
  \item \textbf{Weakening}: Case analysis on assumption membership
  \item \textbf{Modal/temporal rules}: Do not apply (require empty context)
\end{itemize}

\subsection{Consistency}

\begin{definition}[Consistent]
A context $\context$ is \textbf{consistent} if $\context \not\proves \falsum$.
\end{definition}

\begin{definition}[Maximal Consistent]
A context $\context$ is \textbf{maximal consistent} if it is consistent and
for all $\varphi \notin \context$, the context $\varphi :: \context$ is inconsistent.
\end{definition}

\subsection{Completeness}

The completeness proof uses the semantic canonical model construction.
World states are defined as equivalence classes of (history, time) pairs, making the truth lemma straightforward by construction.

\begin{lemma}[Lindenbaum]
Every consistent context can be extended to a maximal consistent context.
This is proven as \texttt{set\_lindenbaum} using Zorn's lemma.
\end{lemma}

\begin{definition}[Semantic World State]
A world state is an equivalence class of (history, time) pairs under the relation where two pairs are equivalent iff they denote the same underlying world state.
This is formalized as \texttt{SemanticWorldState}.
\end{definition}

\begin{definition}[Canonical Frame]
The semantic canonical frame (\texttt{SemanticCanonicalFrame}) has:
\begin{itemize}
  \item World states: Equivalence classes of history-time pairs over maximal consistent sets
  \item Times: Integers ($\mathbb{Z}$)
  \item Task relation: Defined via history existence (\texttt{SemanticTaskRelV2})
\end{itemize}
The frame satisfies nullity and compositionality.
\end{definition}

\begin{definition}[Canonical Valuation]
An atom $p$ is true at world state $[\tau, x]$ iff $p \in \text{mcs}(\tau(x))$.
\end{definition}

\begin{lemma}[Truth Lemma]
In the semantic canonical model, $\varphi \in \mathsf{MCS}(\tau(x))$ iff $\varphi$ is true at world state $[\tau, x]$.\footnote{This is proven as \texttt{semantic\_truth\_lemma\_v2}.}
\end{lemma}

\begin{theorem}[Weak Completeness]
If $\valid{\varphi}$ then $\proves \varphi$.\footnote{This is proven as \texttt{semantic\_weak\_completeness}.}
\end{theorem}

\begin{theorem}[Strong Completeness]
If $\context \satisfies \varphi$ then $\context \proves \varphi$.\footnote{This is proven as \texttt{main\_strong\_completeness} with bridge sorries for the general valid connection.}
\end{theorem}

\begin{theorem}[Provable iff Valid]
$\proves \varphi$ iff $\valid{\varphi}$.\footnote{This is proven as \texttt{main\_provable\_iff\_valid}, establishing the completeness of the proof system.}
\end{theorem}

\begin{definition}[Finite Model Property]
If a formula is satisfiable, it is satisfiable in a finite model bounded by the formula's modal and temporal depth.\footnote{This is stated as \texttt{finite\_model\_property}.}
\end{definition}

\subsection{Proof Strategy}

The semantic approach defines world states as equivalence classes of (history, time) pairs.
Two pairs $(\tau_1, x_1)$ and $(\tau_2, x_2)$ are equivalent iff $\tau_1(x_1) = \tau_2(x_2)$ as underlying world states.

This construction offers key advantages:
\begin{itemize}
  \item \textbf{Truth lemma}: Follows directly from the quotient construction.
  Membership in a maximal consistent set corresponds to truth by definition of the equivalence class.
  \item \textbf{Compositionality}: The task relation is defined via history existence.
  Two world states are related by a duration $d$ iff there exists a history passing through both states at times differing by $d$.
  History concatenation provides the compositionality proof.
  \item \textbf{Negation-completeness}: The semantic approach does not require proving negation-completeness of arbitrary locally consistent sets, a property that caused difficulties in the syntactic approach.
\end{itemize}

The syntactic approach (building world states from negation-complete maximal consistent sets) is superseded.
The deprecated code is archived in \texttt{Boneyard/} for historical reference.

\subsection{Decidability}

The decidability of TM bimodal logic is established via a tableau-based decision procedure.

\begin{theorem}[Decidability]
Validity in TM bimodal logic is decidable: for any formula $\varphi$, either $\valid{\varphi}$ or $\neg\valid{\varphi}$.
\end{theorem}

\begin{theorem}[Decision Soundness]
If the decision procedure returns ``valid'' with proof $\pi$, then $\valid{\varphi}$.
\end{theorem}

The decision procedure operates as follows:
\begin{enumerate}
  \item Try direct axiom proof (fast path optimization)
  \item Try proof search with limited depth
  \item Build tableau starting with signed formula $F(\varphi)$
  \item If all branches close: formula is valid, extract proof
  \item If open saturated branch: formula is invalid, extract countermodel
\end{enumerate}

\subsubsection{Tableau Structure}

The tableau uses \textbf{signed formulas} with annotations:
\begin{itemize}
  \item $T(\varphi)$: formula $\varphi$ is assumed true
  \item $F(\varphi)$: formula $\varphi$ is assumed false
\end{itemize}

\textbf{Expansion rules} are categorized as:
\begin{itemize}
  \item \textbf{Propositional}: $T(\varphi \land \psi)$ splits, $F(\varphi \imp \psi)$ splits, etc.
  \item \textbf{Modal}: $T(\nec\varphi)$ propagates to accessible worlds, $F(\poss\varphi)$ creates witness
  \item \textbf{Temporal}: $T(\always\varphi)$ propagates to future times, $F(\sometimes\varphi)$ creates witness
\end{itemize}

A branch \textbf{closes} when it contains both $T(\varphi)$ and $F(\varphi)$ for some formula.
A branch is \textbf{saturated} when no expansion rules apply.

\subsubsection{Complexity}

\begin{center}
\begin{tabular}{@{}ll@{}}
\toprule
Measure & Complexity \\
\midrule
Time & $O(2^n)$ where $n$ is formula size \\
Space & $O(n)$ \\
Class & PSPACE-complete \\
\bottomrule
\end{tabular}
\end{center}

\subsubsection{Decision Result Types}

The decision procedure returns one of three outcomes:
\begin{itemize}
  \item \texttt{valid proof}: Formula is valid with derivation tree
  \item \texttt{invalid counter}: Formula is invalid with countermodel
  \item \texttt{timeout}: Resources exhausted before decision
\end{itemize}

\subsubsection{Implementation Status}

\begin{center}
\begin{tabular}{@{}lll@{}}
\toprule
Submodule & Status & Notes \\
\midrule
SignedFormula & Complete & Sign, SignedFormula, Branch types \\
Tableau & Complete & Expansion rules \\
Closure & Complete & Branch closure detection \\
Saturation & Complete & Fuel-based termination \\
ProofExtraction & Partial & Axiom instances only \\
CountermodelExtraction & Complete & From open branches \\
DecisionProcedure & Complete & Main decide function \\
Soundness & Proven & \texttt{decide\_sound} \\
Completeness & Partial & Requires Finite Model Property \\
\bottomrule
\end{tabular}
\end{center}

The soundness proof establishes that a ``valid'' result implies semantic validity.
The completeness proof requires the Finite Model Property (FMP), which is stated but not yet fully formalized.

\subsection{Implementation Status}

\begin{center}
\begin{tabular}{@{}lll@{}}
\toprule
Component & Status & Lean \\
\midrule
Soundness & Proven & \texttt{soundness} \\
Deduction Theorem & Proven & \texttt{deduction\_theorem} \\
Lindenbaum Lemma & Proven & \texttt{set\_lindenbaum} \\
Canonical Frame & Proven & \texttt{SemanticCanonicalFrame} \\
Truth Lemma & Proven & \texttt{semantic\_truth\_lemma\_v2} \\
Weak Completeness & Proven & \texttt{semantic\_weak\_completeness} \\
Strong Completeness & Proven* & \texttt{main\_strong\_completeness} \\
Provable iff Valid & Proven & \texttt{main\_provable\_iff\_valid} \\
Finite Model Property & Statement & \texttt{finite\_model\_property} \\
Decidability & Soundness & \texttt{decide\_sound} \\
\bottomrule
\end{tabular}
\end{center}

\noindent
* Strong completeness has bridge sorries for the connection between general validity and frame validity.

\end{document}
