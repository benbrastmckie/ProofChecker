\documentclass[../BimodalReference.tex]{subfiles}
\begin{document}

\section{Metalogic}

The metalogic of the bimodal logic \textbf{TM} establishes the key properties connecting syntax and semantics.
This chapter presents these results with the Representation Theorem as the central result, from which completeness follows as a corollary.

\subsection{Soundness}

The soundness theorem establishes that derivability implies semantic validity.

\begin{theorem}[Soundness]
If $\context \proves \varphi$ then $\context \satisfies \varphi$.
\end{theorem}

The proof proceeds by induction on the derivation structure:
\begin{itemize}
  \item \textbf{Axioms}: Each of the 14 axiom schemata is proven valid
  \item \textbf{Assumptions}: Assumed formulas are true by hypothesis
  \item \textbf{Modus ponens}: Validity preserved under implication elimination
  \item \textbf{Necessitation}: Valid formulas become necessarily valid
  \item \textbf{Temporal necessitation}: Valid formulas become always-future valid
  \item \textbf{Temporal duality}: Past-future swap preserves validity
  \item \textbf{Weakening}: Adding premises preserves semantic consequence
\end{itemize}

\begin{center}
\begin{tabular}{@{}lll@{}}
\toprule
Axiom Validity & Lean Theorem & Technique \\
\midrule
\texttt{prop\_k\_valid} & Propositional K & Propositional reasoning \\
\texttt{prop\_s\_valid} & Propositional S & Propositional reasoning \\
\texttt{ex\_falso\_valid} & EFQ & Vacuous implication \\
\texttt{peirce\_valid} & Peirce & Classical case analysis \\
\texttt{modal\_t\_valid} & MT & Reflexivity of accessibility \\
\texttt{modal\_4\_valid} & M4 & Transitivity of accessibility \\
\texttt{modal\_b\_valid} & MB & Symmetry of accessibility \\
\texttt{modal\_5\_collapse\_valid} & M5 & S5 equivalence structure \\
\texttt{modal\_k\_dist\_valid} & MK & Distribution \\
\texttt{temp\_k\_dist\_valid} & TK & Temporal distribution \\
\texttt{temp\_4\_valid} & T4 & Transitivity of time \\
\texttt{temp\_a\_valid} & TA & Temporal connectedness \\
\texttt{temp\_l\_valid} & TL & Always implies recurrence \\
\texttt{modal\_future\_valid} & MF & Time-shift invariance \\
\texttt{temp\_future\_valid} & TF & Time-shift invariance \\
\bottomrule
\end{tabular}
\end{center}

The MF and TF axioms use time-shift invariance (via \texttt{WorldHistory.time\_shift}) to relate truth at different times.

\subsection{Core Infrastructure}

The completeness proof requires three foundational components: the deduction theorem, maximal consistent sets, and Lindenbaum's lemma.
These provide the infrastructure for constructing canonical models.

\subsubsection{Deduction Theorem}

\begin{theorem}[Deduction Theorem]
If $A :: \context \proves B$ then $\context \proves A \imp B$.
\end{theorem}

The proof uses well-founded induction on derivation height, handling each rule:
\begin{itemize}
  \item \textbf{Axiom}: Use S axiom to weaken
  \item \textbf{Assumption}: Identity if same, S axiom if different
  \item \textbf{Modus ponens}: Use K axiom distribution
  \item \textbf{Weakening}: Case analysis on assumption membership
  \item \textbf{Modal/temporal rules}: Do not apply (require empty context)
\end{itemize}

\subsubsection{Consistency}

\begin{definition}[Consistent]
A context $\context$ is \textbf{consistent} if $\context \not\proves \falsum$.
\end{definition}

\begin{definition}[Maximal Consistent]
A context $\context$ is \textbf{maximal consistent} if it is consistent and
for all $\varphi \notin \context$, the context $\varphi :: \context$ is inconsistent.
\end{definition}

Maximal consistent sets (MCS) have the key property that they are ``negation-complete'':
for every formula $\varphi$, exactly one of $\varphi$ or $\lneg\varphi$ is in the set.
This property is essential for defining canonical world states.

\subsubsection{Lindenbaum's Lemma}

\begin{lemma}[Lindenbaum]
Every consistent context can be extended to a maximal consistent context.\footnote{This is proven as \texttt{set\_lindenbaum} using Zorn's lemma.}
\end{lemma}

The proof applies Zorn's lemma to the partially ordered set of consistent supersets of the given context.
The key step is showing that the union of any chain of consistent sets is itself consistent.
This follows because any derivation uses only finitely many premises, so a derivation of $\falsum$ from the union would have to come from some finite subset, which is contained in some member of the chain---contradicting that member's consistency.

\subsection{Representation Theory}

The Representation Theorem is the central result of the metalogic, establishing the fundamental bridge between syntactic consistency and semantic satisfiability.
The subsequent completeness theorems follow directly from this result.

\subsubsection{Canonical World States}

A \textbf{canonical world state} is derived from a maximal consistent set.
The semantic approach defines world states as equivalence classes of (history, time) pairs.

\begin{definition}[Semantic World State]
A world state is an equivalence class of (history, time) pairs under the relation where two pairs are equivalent iff they denote the same underlying world state.\footnote{This is formalized as \texttt{SemanticWorldState}.}
\end{definition}

\begin{definition}[Canonical Frame]
The \texttt{SemanticCanonicalFrame} has:
\begin{itemize}
  \item World states: Equivalence classes of history-time pairs over maximal consistent sets
  \item Times: Integers ($\mathbb{Z}$)
  \item Task relation: Defined via history existence (\texttt{SemanticTaskRelV2})
\end{itemize}
The frame satisfies nullity and compositionality.
\end{definition}

\begin{definition}[Canonical Valuation]
An atom $p$ is true at world state $[\tau, x]$ iff $p \in \mathsf{MCS}(\tau(x))$.
\end{definition}

\begin{lemma}[Truth Lemma]
In the semantic canonical model, $\varphi \in \mathsf{MCS}(\tau(x))$ iff $\varphi$ is true at world state $[\tau, x]$.\footnote{This is proven as \texttt{semantic\_truth\_lemma\_v2}.}
\end{lemma}

The truth lemma follows directly from the quotient construction.
Membership in a maximal consistent set corresponds to truth by definition of the equivalence class.

\subsubsection{Representation Theorem}

\begin{theorem}[Representation Theorem]
Every consistent context is satisfiable in the canonical model.\footnote{This is proven as \texttt{representation\_theorem}.}
\end{theorem}

This theorem is the pivotal result linking syntax to semantics.
The proof strategy is:
\begin{enumerate}
  \item Given a consistent context $\context$, convert it to a set $S = \texttt{contextToSet}(\context)$.
  \item Apply Lindenbaum's lemma to extend $S$ to a maximal consistent set $M$.
  \item View $M$ as a canonical world state.
  \item By the truth lemma, all formulas in $\context$ are satisfied at this world.
\end{enumerate}

The elegance of this approach is that the MCS construction makes the truth lemma essentially trivial---truth \emph{is} membership in the MCS.

\begin{theorem}[Strong Representation Theorem]
If $\context \not\proves \varphi$, then $\context \cup \{\lneg\varphi\}$ is satisfiable in the canonical model.\footnote{This is proven as \texttt{strong\_representation\_theorem}.}
\end{theorem}

\subsection{Completeness as Corollary}

The completeness theorems follow directly from the Representation Theorem via contrapositive arguments.
This demonstrates the power of the representation-first approach.

\subsubsection{Weak Completeness}

\begin{theorem}[Weak Completeness]
If $\valid{\varphi}$ then $\proves \varphi$.\footnote{This is proven as \texttt{semantic\_weak\_completeness}.}
\end{theorem}

The proof proceeds by contraposition:
\begin{enumerate}
  \item Assume $\not\proves \varphi$ (i.e., the empty context does not derive $\varphi$).
  \item Then $\{\lneg\varphi\}$ is consistent (otherwise we could derive $\varphi$).
  \item By the Representation Theorem, $\{\lneg\varphi\}$ is satisfiable in the canonical model.
  \item So there exists a world where $\lneg\varphi$ is true, meaning $\varphi$ is false.
  \item Hence $\varphi$ is not valid.
\end{enumerate}
By contraposition, validity implies provability.

\subsubsection{Strong Completeness}

\begin{theorem}[Strong Completeness]
If $\context \satisfies \varphi$ then $\context \proves \varphi$.\footnote{This is proven as \texttt{main\_strong\_completeness} with bridge sorries for the generalization.}
\end{theorem}

The proof extends weak completeness using an implication chain technique:
\begin{enumerate}
  \item Assume semantic consequence: $\context \satisfies \varphi$.
  \item For context $\context = [\psi_1, \ldots, \psi_n]$, build the implication chain $\psi_1 \imp (\psi_2 \imp \cdots (\psi_n \imp \varphi))$.
  \item Show this chain is valid (from the semantic consequence assumption).
  \item By weak completeness, the chain is provable.
  \item Unfold the chain with repeated modus ponens applications to obtain $\context \proves \varphi$.
\end{enumerate}

\begin{theorem}[Provable iff Valid]
$\proves \varphi$ iff $\valid{\varphi}$.\footnote{This is proven as \texttt{main\_provable\_iff\_valid}, establishing completeness of \textbf{TM}.}
\end{theorem}

\subsubsection{Two Canonical Model Approaches}

The codebase contains two canonical model constructions.
Understanding their differences explains why the semantic approach is primary.

\paragraph{Syntactic Approach.}
World states are directly identified with maximal consistent sets.
Accessibility is defined via modal witnesses: $\nec\varphi \in w$ implies $\varphi \in w'$ for all accessible $w'$.
This approach requires explicit negation-completeness proofs for locally consistent sets.
The syntactic approach is archived in \texttt{Boneyard/} for historical reference.

\paragraph{Semantic Approach.}
World states are equivalence classes of (history, time) pairs, where two pairs are equivalent iff they denote the same underlying world state.
This approach offers key advantages:
\begin{itemize}
  \item \textbf{Truth lemma}: Follows trivially from the quotient construction.
  \item \textbf{Compositionality}: The task relation is defined via history concatenation, making compositionality proofs straightforward.
  \item \textbf{Negation-completeness}: The semantic approach does not require proving negation-completeness of arbitrary locally consistent sets, a property that caused difficulties in the syntactic approach.
\end{itemize}

\subsection{Finite Model Property}

\begin{definition}[Finite Model Property]
If a formula is satisfiable, it is satisfiable in a finite model bounded by the formula's modal and temporal depth.\footnote{This is stated as \texttt{finite\_model\_property}.}
\end{definition}

The finite model property connects the representation infrastructure to decidability.
The bound on model size is $2^{|\mathsf{closure}(\varphi)|}$, where the subformula closure contains all relevant formulas for determining the truth of $\varphi$.

\subsection{Decidability}

The decidability of TM bimodal logic is established via a tableau-based decision procedure.

\begin{theorem}[Decidability]
Validity in TM bimodal logic is decidable: for any formula $\varphi$, either $\valid{\varphi}$ or $\neg\valid{\varphi}$.
\end{theorem}

\begin{theorem}[Decision Soundness]
If the decision procedure returns ``valid'' with proof $\pi$, then $\valid{\varphi}$.\footnote{This is proven as \texttt{decide\_sound}.}
\end{theorem}

The decision procedure operates as follows:
\begin{enumerate}
  \item Try direct axiom proof (fast path optimization)
  \item Try proof search with limited depth
  \item Build tableau starting with signed formula $F(\varphi)$
  \item If all branches close: formula is valid, extract proof
  \item If open saturated branch: formula is invalid, extract countermodel
\end{enumerate}

\subsubsection{Tableau Structure}

The tableau uses \textbf{signed formulas} with annotations:
\begin{itemize}
  \item $T(\varphi)$: formula $\varphi$ is assumed true
  \item $F(\varphi)$: formula $\varphi$ is assumed false
\end{itemize}

\textbf{Expansion rules} are categorized as:
\begin{itemize}
  \item \textbf{Propositional}: $T(\varphi \land \psi)$ splits, $F(\varphi \imp \psi)$ splits, etc.
  \item \textbf{Modal}: $T(\nec\varphi)$ propagates to accessible worlds, $F(\poss\varphi)$ creates witness
  \item \textbf{Temporal}: $T(\always\varphi)$ propagates to future times, $F(\sometimes\varphi)$ creates witness
\end{itemize}

A branch \textbf{closes} when it contains both $T(\varphi)$ and $F(\varphi)$ for some formula.
A branch is \textbf{saturated} when no expansion rules apply.

\subsubsection{Complexity}

\begin{center}
\begin{tabular}{@{}ll@{}}
\toprule
Measure & Complexity \\
\midrule
Time & $O(2^n)$ where $n$ is formula size \\
Space & $O(n)$ \\
Class & PSPACE-complete \\
\bottomrule
\end{tabular}
\end{center}

\subsubsection{Decision Result Types}

The decision procedure returns one of three outcomes:
\begin{itemize}
  \item \texttt{valid proof}: Formula is valid with derivation tree
  \item \texttt{invalid counter}: Formula is invalid with countermodel
  \item \texttt{timeout}: Resources exhausted before decision
\end{itemize}

\subsection{Implementation Status}

\subsubsection{Decidability Implementation}

\begin{center}
\begin{tabular}{@{}lll@{}}
\toprule
Submodule & Status & Notes \\
\midrule
SignedFormula & Complete & Sign, SignedFormula, Branch types \\
Tableau & Complete & Expansion rules \\
Closure & Complete & Branch closure detection \\
Saturation & Complete & Fuel-based termination \\
ProofExtraction & Partial & Axiom instances only \\
CountermodelExtraction & Complete & From open branches \\
DecisionProcedure & Complete & Main decide function \\
Soundness & Proven & \texttt{decide\_sound} \\
Completeness & Partial & Requires Finite Model Property \\
\bottomrule
\end{tabular}
\end{center}

\subsubsection{Metalogic Implementation}

\begin{center}
\begin{tabular}{@{}lll@{}}
\toprule
Component & Status & Lean \\
\midrule
Soundness & Proven & \texttt{soundness} \\
Deduction Theorem & Proven & \texttt{deduction\_theorem} \\
Lindenbaum Lemma & Proven & \texttt{set\_lindenbaum} \\
Canonical Frame & Proven & \texttt{SemanticCanonicalFrame} \\
Truth Lemma & Proven & \texttt{semantic\_truth\_lemma\_v2} \\
Representation Theorem & Proven & \texttt{representation\_theorem} \\
Weak Completeness & Proven & \texttt{semantic\_weak\_completeness} \\
Strong Completeness & Proven* & \texttt{main\_strong\_completeness} \\
Provable iff Valid & Proven & \texttt{main\_provable\_iff\_valid} \\
Finite Model Property & Statement & \texttt{finite\_model\_property} \\
Decidability Soundness & Proven & \texttt{decide\_sound} \\
\bottomrule
\end{tabular}
\end{center}

\noindent
* Strong completeness has bridge sorries for the connection between general validity and frame validity.

\end{document}
