\documentclass[../BimodalReference.tex]{subfiles}
\begin{document}

\section{Theorems}

\subsection{Perpetuity Principles}

The perpetuity principles establish deep connections between modal necessity ($\nec$) and temporal operators ($\always$, $\sometimes$).

\begin{theorem}[P1: Necessity Implies Always]
$\proves \nec\varphi \imp \always\varphi$
\end{theorem}

\begin{theorem}[P2: Sometimes Implies Possible]
$\proves \sometimes\varphi \imp \poss\varphi$
\end{theorem}

\begin{theorem}[P3: Necessity of Perpetuity]
$\proves \nec\varphi \imp \nec\always\varphi$
\end{theorem}

\begin{theorem}[P4: Possibility of Occurrence]
$\proves \poss\sometimes\varphi \imp \poss\varphi$
\end{theorem}

\begin{theorem}[P5: Persistent Possibility]
$\proves \poss\sometimes\varphi \imp \always\poss\varphi$
\end{theorem}

\begin{theorem}[P6: Occurrent Necessity is Perpetual]
$\proves \sometimes\nec\varphi \imp \nec\always\varphi$
\end{theorem}

All six perpetuity principles are fully proven in the Lean implementation.

\begin{center}
\begin{tabular}{@{}lll@{}}
\toprule
Principle & Lean Theorem & Key Lemmas \\
\midrule
P1 & \texttt{perpetuity\_1} & MF, TF, MT \\
P2 & \texttt{perpetuity\_2} & Contraposition of P1 \\
P3 & \texttt{perpetuity\_3} & P1, box\_mono \\
P4 & \texttt{perpetuity\_4} & Contraposition \\
P5 & \texttt{perpetuity\_5} & modal\_5, temporal K \\
P6 & \texttt{perpetuity\_6} & P5, bridge lemmas \\
\bottomrule
\end{tabular}
\end{center}

\subsection{Modal S5 Theorems}

\begin{theorem}[T-Box-to-Diamond]
$\proves \nec\varphi \imp \poss\varphi$
\end{theorem}

\begin{theorem}[Box Distributes Over Disjunction]
$\proves (\nec\varphi \lor \nec\psi) \imp \nec(\varphi \lor \psi)$
\end{theorem}

\begin{theorem}[Box Preserves Contraposition]
$\proves \nec(\varphi \imp \psi) \imp \nec(\lneg\psi \imp \lneg\varphi)$
\end{theorem}

\begin{theorem}[K Distribution for Diamond]
$\proves \nec(\varphi \imp \psi) \imp (\poss\varphi \imp \poss\psi)$
\end{theorem}

\begin{theorem}[S5 Collapse]
$\proves \poss\nec\varphi \leftrightarrow \nec\varphi$
\end{theorem}

\begin{theorem}[Box-Conjunction Biconditional]
$\proves \nec(\varphi \land \psi) \leftrightarrow (\nec\varphi \land \nec\psi)$
\end{theorem}

\begin{theorem}[Diamond-Disjunction Biconditional]
$\proves \poss(\varphi \lor \psi) \leftrightarrow (\poss\varphi \lor \poss\psi)$
\end{theorem}

\subsection{Modal S4 Properties}

The following S4 properties are derived from the TM axiom system.

\begin{theorem}[Modal 5]
$\proves \poss\varphi \imp \nec\poss\varphi$
\end{theorem}

\begin{theorem}[Diamond 4]
$\proves \poss\poss\varphi \imp \poss\varphi$
\end{theorem}

\begin{theorem}[Box Monotonicity]
If $\proves \varphi \imp \psi$ then $\proves \nec\varphi \imp \nec\psi$.
\end{theorem}

\begin{theorem}[Diamond Monotonicity]
If $\proves \varphi \imp \psi$ then $\proves \poss\varphi \imp \poss\psi$.
\end{theorem}

\subsection{Propositional Theorems}

\begin{theorem}[Identity]
$\proves \varphi \imp \varphi$
\end{theorem}

\begin{theorem}[Double Negation Introduction]
$\proves \varphi \imp \lneg\lneg\varphi$
\end{theorem}

\begin{theorem}[Double Negation Elimination]
$\proves \lneg\lneg\varphi \imp \varphi$
\end{theorem}

\begin{theorem}[Contraposition]
If $\proves \varphi \imp \psi$ then $\proves \lneg\psi \imp \lneg\varphi$.
\end{theorem}

\begin{theorem}[De Morgan (Disjunction)]
$\proves \lneg(\varphi \lor \psi) \leftrightarrow (\lneg\varphi \land \lneg\psi)$
\end{theorem}

\begin{theorem}[De Morgan (Conjunction)]
$\proves \lneg(\varphi \land \psi) \leftrightarrow (\lneg\varphi \lor \lneg\psi)$
\end{theorem}

\subsection{Combinator Infrastructure}

The combinator infrastructure provides Hilbert-style proof tools.

\begin{theorem}[B Combinator]
$\proves (B \imp C) \imp ((A \imp B) \imp (A \imp C))$
\end{theorem}

\begin{theorem}[Implication Transitivity]
If $\proves A \imp B$ and $\proves B \imp C$ then $\proves A \imp C$.
\end{theorem}

\begin{theorem}[Pairing]
$\proves A \imp (B \imp (A \land B))$
\end{theorem}

\begin{theorem}[Classical Merge]
$\proves (P \imp Q) \imp ((\lneg P \imp Q) \imp Q)$
\end{theorem}

\subsection{Generalized Necessitation}

\begin{theorem}[Generalized Modal Necessitation]
If $\context \proves \varphi$ then $\nec\context \proves \nec\varphi$
where $\nec\context = [\nec\psi \mid \psi \in \context]$.
\end{theorem}

\begin{theorem}[Generalized Temporal Necessitation]
If $\context \proves \varphi$ then $\allfuture\context \proves \allfuture\varphi$
where $\allfuture\context = [\allfuture\psi \mid \psi \in \context]$.
\end{theorem}

\subsection{Module Organization}

\begin{center}
\begin{tabular}{@{}ll@{}}
\toprule
Module & Contents \\
\midrule
\texttt{Perpetuity.lean} & P1-P6 principles \\
\texttt{ModalS5.lean} & S5 characteristic theorems \\
\texttt{ModalS4.lean} & S4 properties (modal\_5, diamond\_4) \\
\texttt{Propositional.lean} & Classical propositional theorems \\
\texttt{Combinators.lean} & B, I, S combinators, imp\_trans \\
\texttt{GeneralizedNecessitation.lean} & Context-level necessitation \\
\bottomrule
\end{tabular}
\end{center}

\end{document}
