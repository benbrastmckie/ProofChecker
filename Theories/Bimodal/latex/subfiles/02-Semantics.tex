\documentclass[../BimodalReference.tex]{subfiles}
\begin{document}

\section{Task Semantics}

\subsection{Task Frames}

Task frames are the fundamental semantic structures for \textbf{TM}.
They abstract from universal laws governing transitions between world states while still retaining the temporal duration for a transition to complete.

% TODO: I don't like how this goes below. Instead I want to introduce the primitives first (the first bullet points) but using a table instead of bullet points to characterize these primitive types. Only then should the definition be given using the types that have been introduced in this table, where the definition can then be asserted somewhat more directly by defining the type of the task frame and imposing the constraints. It is also important that the constraints respect types (they types should not be omitted anywhere there is quantification throughout this document).

\begin{definition}[Task Frame]
A \textbf{task frame} over temporal type $T$ is a triple $\taskframe = (\worldstate, T, {\taskto{}})$ where:
\begin{itemize}
  \item $\worldstate$ is the \textbf{world state} type where these are understood to be complete instantaneous total states of the system in question
  \item $T$ is the \textbf{temporal duration} type with totally ordered abelian group structure for adding and subtracting times
  \item $w \taskto{x} u$ is the \textbf{task relation} type, indicating that it is possible for world state $w$ to transition to $u$ in duration $x$
\end{itemize}
satisfying two constraints:
\begin{enumerate}
  \item \textbf{Nullity}: $\forall w.\, w \taskto{0} w$
  \item \textbf{Compositionality}: $\forall w\, u\, v\, x\, y.\, w \taskto{x} u \land u \taskto{y} v \to w \taskto{x+y} v$
\end{enumerate}
\end{definition}

\noindent
Nullity ensures that zero-duration tasks leave the world state unchanged.
Compositionality ensures that executing tasks sequentially yields results consistent with a single task of combined duration.

% TODO: so long as the types for the primitives are provided, this table below can be omitted

\begin{center}
\begin{tabular}{@{}ll@{}}
\toprule
Lean Field & Type \\
\midrule
\texttt{WorldState} & \texttt{Type} \\
\leanTaskRel & \texttt{WorldState $\to$ T $\to$ WorldState $\to$ Prop} \\
\leanNullity & \texttt{$\forall$ w, task\_rel w 0 w} \\
\leanCompositionality & \texttt{$\forall$ w u v x y, task\_rel w x u $\to$ task\_rel u y v $\to$ task\_rel w (x+y) v} \\
\bottomrule
\end{tabular}
\end{center}

\subsection{World Histories}

A world history is a function from times to world states that respects the task relation over a convex temporal domain.
World histories represent possible paths through the space of world states.

\begin{definition}[Convex Domain]
A domain $\domain : T \to \text{Prop}$ is \textbf{convex} if whenever $a, c \in \domain$ with $a \le c$, every time $b$ with $a \le b \le c$ is also in $\domain$:
\[
\forall a\, b\, c.\, \domain(a) \land \domain(c) \land a \le b \land b \le c \to \domain(b)
\]
Convexity ensures the domain has no temporal gaps.
\end{definition}

\begin{definition}[Respects Task]
A function $\tau : \domain \to \worldstate$ \textbf{respects the task relation} if for all $s, t \in \domain$ with $s \le t$:
\[
\tau(s) \taskto{t - s} \tau(t)
\]
This ensures the history is consistent with possible task executions.
\end{definition}

\begin{definition}[World History]
A \textbf{world history} in task frame $\taskframe$ is a structure $\tau$ with:
\begin{itemize}
  \item $\domain : T \to \text{Prop}$ --- a convex temporal domain
  \item $\tau : (t : T) \to \domain(t) \to \worldstate$ --- world state assignment
  \item $\tau$ respects the task relation
\end{itemize}
We write $\histories_{\taskframe}$ for the set of all world histories over frame $\taskframe$.
\end{definition}

\begin{center}
\begin{tabular}{@{}ll@{}}
\toprule
Lean Field & Type \\
\midrule
\leanDomain & \texttt{T $\to$ Prop} \\
\leanConvex & \texttt{$\forall$ a b c, domain a $\to$ domain c $\to$ a $\le$ b $\to$ b $\le$ c $\to$ domain b} \\
\leanStates & \texttt{(t : T) $\to$ domain t $\to$ WorldState} \\
\leanRespTask & \texttt{$\forall$ s t (hs : domain s) (ht : domain t), s $\le$ t $\to$ task\_rel (states s hs) (t - s) (states t ht)} \\
\bottomrule
\end{tabular}
\end{center}

\subsection{Task Models}

A task model extends a task frame with an interpretation function that assigns truth values to atomic propositions at world states.

Atomic propositions (sentence letters) express \textbf{propositions}---instantaneous ways for the system to be.
A proposition can be realized by zero or more world states, unlike world states themselves which are unique configurations.

\begin{definition}[Task Model]
A \textbf{task model} over frame $\taskframe$ is a structure $\model = (\taskframe, I)$ where the \textbf{interpretation function}:
\[
I : \worldstate \to \text{String} \to \text{Prop}
\]
assigns truth values to atomic propositions at each world state.
We write $I(w, p)$ or simply $w \vDash p$ when $p$ is true at world state $w$.
\end{definition}

\subsection{Truth Conditions}

Truth is evaluated relative to three contextual parameters: a model $\model$ providing the interpretation, a world history $\tau$ representing a possible temporal path, and a time $t$ within that history's domain.
Together these parameters determine the truth value of any formula.

\begin{definition}[Truth at a Point]
For model $\model$, history $\tau$, time $t \in \domain(\tau)$:
\begin{align*}
  \model, \tau, t \vDash p &\Iff I(\tau(t), p) \\
  \model, \tau, t \nvDash \falsum \\
  \model, \tau, t \vDash \varphi \imp \psi &\Iff
    \model, \tau, t \nvDash \varphi \text{ or } \model, \tau, t \vDash \psi \\
  \model, \tau, t \vDash \nec\varphi &\Iff
    \model, \sigma, t \vDash \varphi \text{ for all } \sigma \in \histories_{\taskframe} \\
  \model, \tau, t \vDash \allpast\varphi &\Iff
    \model, \tau, s \vDash \varphi \text{ for all } s \in \domain(\tau) \text{ where } s < t \\
  \model, \tau, t \vDash \allfuture\varphi &\Iff
    \model, \tau, s \vDash \varphi \text{ for all } s \in \domain(\tau) \text{ where } t < s
\end{align*}
\end{definition}

\noindent
The modal operator $\nec$ quantifies over all world histories $\sigma$ in $\histories_{\taskframe}$ at the current time $t$.
The temporal operators $\allpast$ and $\allfuture$ quantify over earlier and later times within the same history's domain.

\textit{Note:} The Lean implementation restricts temporal quantification to times in the history's domain $\domain(\tau)$, matching the semantic clauses above.
The source paper quantifies over all times $s \in T$; this design choice ensures well-defined evaluation when histories have bounded domains.

\subsection{Time-Shift}

The time-shift operation translates a history by a temporal offset, enabling us to relate truth at different times.

\begin{definition}[Time-Shift]
Given history $\tau$ and offset $\Delta : T$, the \textbf{time-shifted history} $\leanTimeShift(\tau, \Delta)$ is defined by:
\begin{itemize}
  \item $\domain_{\tau'}(t) \Iff \domain_\tau(t + \Delta)$
  \item $\tau'(t) = \tau(t + \Delta)$
\end{itemize}
\end{definition}

\noindent
Time-shifting preserves the essential structure of histories:

\begin{theorem}[Convexity Preservation]
If $\tau$ has convex domain, so does $\leanTimeShift(\tau, \Delta)$.
\end{theorem}

\begin{theorem}[Task Respect Preservation]
If $\tau$ respects the task relation, so does $\leanTimeShift(\tau, \Delta)$.
\end{theorem}

\noindent
The time-shift construction is fundamental for proving the temporal axioms MF and TF.
It establishes a bijection between histories at different times, allowing us to transfer truth values across temporal contexts.

\subsection{Logical Consequence and Validity}

Validity and logical consequence are defined uniformly across all temporal types, frames, models, histories, and times.

\begin{definition}[Validity]
A formula $\varphi$ is \textbf{valid} (written $\models \varphi$) if it is true at every model-history-time triple:
\[
\forall T.\, \forall \taskframe : \text{TaskFrame}\, T.\, \forall \model : \text{TaskModel}\, \taskframe.\,
\forall \tau \in \histories_{\taskframe}.\, \forall t \in \domain(\tau).\,
\model, \tau, t \vDash \varphi
\]
where $T$ ranges over all types with ordered additive group structure.
\end{definition}

\noindent
Validity ensures truth across all possible semantic configurations, guaranteeing logical necessity.

\begin{definition}[Logical Consequence]
A formula $\varphi$ is a \textbf{logical consequence} of context $\context$ (written $\context \models \varphi$) if:
whenever all formulas in $\context$ are true at some model-history-time triple, $\varphi$ is also true there.
\end{definition}

\begin{definition}[Satisfiability]
A context $\context$ is \textbf{satisfiable} in temporal type $T$ if there exist a frame $\taskframe$, model $\model$, history $\tau$, and time $t$ such that all formulas in $\context$ are true at $\model, \tau, t$.
\end{definition}

\begin{theorem}[Monotonicity]
Logical consequence is monotonic: if $\context \subseteq \Delta$ and $\context \models \varphi$, then $\Delta \models \varphi$.
\end{theorem}

\end{document}
