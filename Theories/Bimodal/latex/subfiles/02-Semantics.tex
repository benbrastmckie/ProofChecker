\documentclass[../BimodalReference.tex]{subfiles}
\begin{document}

\section{Task Semantics}

\subsection{Task Frames}

Task frames are the fundamental semantic structures for \textbf{TM}.
They abstract from universal laws governing transitions between world states while still retaining the temporal duration for a transition to complete.

% TODO: I don't like how this goes below. Instead I want to introduce the primitives first (the first bullet points) but using a table instead of bullet points to characterize these primitive types. Only then should the definition be given using the types that have been introduced in this table, where the definition can then be asserted somewhat more directly by defining the type of the task frame and imposing the constraints. It is also important that the constraints respect types (they types should not be omitted anywhere there is quantification throughout this document).

\begin{definition}[Task Frame]
A \textbf{task frame} over temporal type $T$ is a triple $\taskframe = (\worldstate, T, {\taskto{}})$ where:
\begin{itemize}
  \item $\worldstate$ is the \textbf{world state} type where these are understood to be complete instantaneous total states of the system in question
  \item $T$ is the \textbf{temporal duration} type with totally ordered abelian group structure for adding and subtracting times
  \item $w \taskto{x} u$ is the \textbf{task relation} type, indicating that it is possible for world state $w$ to transition to $u$ in duration $x$
\end{itemize}
satisfying two constraints:
\begin{enumerate}
  \item \textbf{Nullity}: $\forall w.\, w \taskto{0} w$
  \item \textbf{Compositionality}: $\forall w\, u\, v\, x\, y.\, w \taskto{x} u \land u \taskto{y} v \to w \taskto{x+y} v$
\end{enumerate}
\end{definition}

\noindent
Nullity ensures that zero-duration tasks leave the world state unchanged.
Compositionality ensures that executing tasks sequentially yields results consistent with a single task of combined duration.

% TODO: so long as the types for the primitives are provided, this table below can be omitted

\begin{center}
\begin{tabular}{@{}ll@{}}
\toprule
Lean Field & Type \\
\midrule
\texttt{WorldState} & \texttt{Type} \\
\leanTaskRel & \texttt{WorldState $\to$ T $\to$ WorldState $\to$ Prop} \\
\leanNullity & \texttt{$\forall$ w, task\_rel w 0 w} \\
\leanCompositionality & \texttt{$\forall$ w u v x y, task\_rel w x u $\to$ task\_rel u y v $\to$ task\_rel w (x+y) v} \\
\bottomrule
\end{tabular}
\end{center}

\subsection{World Histories}

A world history is a function from times to world states that respects the task relation over a convex temporal domain.
World histories represent possible paths through the space of world states.

\begin{definition}[Convex Domain]
A domain $\domain : T \to \text{Prop}$ is \textbf{convex} if whenever $a, c \in \domain$ with $a \le c$, every time $b$ /with $a \le b \le c$ is also in $\domain$:
\[
\forall a\, b\, c.\, \domain(a) \land \domain(c) \land a \le b \land b \le c \to \domain(b)
\]
Convexity ensures the domain has no temporal gaps.
\end{definition}

% TODO: replace 'respects the task relation' with 'lawfull', fixing the grammar accordingly

\begin{definition}[Respects Task]
A function $\tau : \domain \to \worldstate$ \textbf{respects the task relation} if for all $s, t \in \domain$ with $s \le t$:
\[
\tau(s) \taskto{t - s} \tau(t)
\]
This ensures the history is consistent with possible task executions.
\end{definition}

% TODO: don't use the dashes below, use simple sentences that state each constraint in a natural and direct way

% TODO: don't say structure, say what the type of \tau is where it has domain dom that is convex and is lawful.

% TODO: fold the definition above into the definition below, avoiding the need for an extra term 'lawful' by simply stating the lawful constraint in the definition of a world history. Despite using types, this should follow the definition in line 868 in /home/benjamin/Projects/Philosophy/Papers/PossibleWorlds/JPL/possible_worlds.tex

\begin{definition}[World History]
A \textbf{world history} in a task frame $\taskframe$ is a structure $\tau$ that satisfies the following constraints:
\begin{itemize}
  \item $\domain : T \to \text{Prop}$ --- a convex temporal domain
  \item $\tau : (t : T) \to \domain(t) \to \worldstate$ --- world state assignment
  \item $\tau$ respects the task relation
\end{itemize}
We write $\histories_{\taskframe}$ for the set of all world histories over frame $\taskframe$.
\end{definition}

% TODO: this next table can be removed

\begin{center}
\begin{tabular}{@{}ll@{}}
\toprule
Lean Field & Type \\
\midrule
\leanDomain & \texttt{T $\to$ Prop} \\
\leanConvex & \texttt{$\forall$ a b c, domain a $\to$ domain c $\to$ a $\le$ b $\to$ b $\le$ c $\to$ domain b} \\
\leanStates & \texttt{(t : T) $\to$ domain t $\to$ WorldState} \\
\leanRespTask & \texttt{$\forall$ s t (hs : domain s) (ht : domain t), s $\le$ t $\to$ task\_rel (states s hs) (t - s) (states t ht)} \\
\bottomrule
\end{tabular}
\end{center}

\subsection{Task Models}

% TODO: use 'sentence letters' in place of 'atomic propositions' throughout. Never use 'atomic proposition' anywhere.

A task model extends a task frame with an interpretation function that assigns truth values to atomic propositions at world states.

% TODO: define propositions as sets of world states which represent instantaneous ways for a syste to be.

% TODO: don't use dashes. use natural sentences to speak directly.

Atomic propositions (sentence letters) express \textbf{propositions}---instantaneous ways for the system to be that can be realized by zero or more world states.
World states themselves are specific configurations of the total system at an instant.

% TODO: specify types explicitly when stating definitions

\begin{definition}[Task Model]
A \textbf{task model} over frame $\taskframe$ is a structure $\model = (\taskframe, I)$ where the \textbf{interpretation function}:
\[
I : \worldstate \to \text{String} \to \text{Prop}
\]
assigns truth values to atomic propositions at each world state.
We write $I(w, p)$ when $p$ is true at world state $w$.
\end{definition}

\subsection{Truth Conditions}

% TODO: use 'x', 'y', 'z', etc., for times. Also, the time in the definition below is NOT restricted to the domain but may be ANY time in the frame. The same is true in the clauses for the tense operators.

Truth is evaluated relative to three contextual parameters: a model $\model$ providing the interpretation, a world history $\tau$ representing a possible temporal path, and a time $t$ within that history's domain.
Together these parameters determine the truth value of any formula.

\begin{definition}[Truth at a Point]
For model $\model$, history $\tau$, time $t \in \domain(\tau)$:
\begin{align*}
  \model, \tau, t \vDash p &\Iff I(\tau(t), p) \\
  \model, \tau, t \nvDash \falsum \\
  \model, \tau, t \vDash \varphi \imp \psi &\Iff
    \model, \tau, t \nvDash \varphi \text{ or } \model, \tau, t \vDash \psi \\
  \model, \tau, t \vDash \nec\varphi &\Iff
    \model, \sigma, t \vDash \varphi \text{ for all } \sigma \in \histories_{\taskframe} \\
  \model, \tau, t \vDash \allpast\varphi &\Iff
    \model, \tau, s \vDash \varphi \text{ for all } s \in \domain(\tau) \text{ where } s < t \\
  \model, \tau, t \vDash \allfuture\varphi &\Iff
    \model, \tau, s \vDash \varphi \text{ for all } s \in \domain(\tau) \text{ where } t < s
\end{align*}
\end{definition}

\noindent
The modal operator $\nec$ quantifies over all world histories $\sigma$ in $\histories_{\taskframe}$ at the current time $t$.
The temporal operators $\allpast$ and $\allfuture$ quantify over earlier and later times within the same history's domain.


\subsection{Time-Shift}

% TODO: relating truths at different times is not the reason this has been defined. Rather it is to establish the perpetuity theorems. Say this, and briefly sketch what these theorems say, and why they are plausible. See line 436 in /home/benjamin/Projects/Philosophy/Papers/PossibleWorlds/JPL/possible_worlds.tex for simple motivation.

The time-shift operation translates a history by a temporal offset.
This notion plays a critical role in proving the temporal axioms MF and TF.

% TODO: I don't like the informal 'Given a history \tau...' and instead want consistent type notation to be used to get readers accustomed to this way of setting things out. This needs to be maintained uniformly everywhere throughout these latex subfiles documents.

% TODO: the offset should be 'x' which is a duration, not a upper case greek letter! Maintain consistent notation conventions. 

% TODO: use the notation used in def:time-shift-histories from line 1877 in /home/benjamin/Projects/Philosophy/Papers/PossibleWorlds/JPL/possible_worlds.tex

\begin{definition}[Time-Shift]
Given history $\tau$ and offset $\Delta : T$, the \textbf{time-shifted history} $\leanTimeShift(\tau, \Delta)$ is defined by:
\begin{itemize}
  \item $\domain_{\tau'}(t) \Iff \domain_\tau(t + \Delta)$
  \item $\tau'(t) = \tau(t + \Delta)$
\end{itemize}
\end{definition}

\noindent
Time-shifting preserves the essential structure of histories:

% TODO: the \leanTimeShift looks really bad, like kids handwriting. use the notation used in def:time-shift-histories and state these definitions more formally to match the paper and lean implementation

\begin{theorem}[Convexity Preservation]
If $\tau$ has a convex domain, so does $\leanTimeShift(\tau, \Delta)$.
\end{theorem}

\begin{theorem}[Task Preservation]
If $\tau$ respects the task relation, so does $\leanTimeShift(\tau, \Delta)$.
\end{theorem}

\subsection{Logical Consequence and Validity}

Logical consequence and validity are defined uniformly across all temporal types, frames, models, histories, and times.

\begin{definition}[Logical Consequence]
A formula $\varphi$ is a \textbf{logical consequence} of context $\context$ (written $\context \models \varphi$) if:
whenever all formulas in $\context$ are true at some model-history-time triple, $\varphi$ is also true there.
\end{definition}

\begin{definition}[Satisfiability]
A context $\context$ is \textbf{satisfiable} in temporal type $T$ if there exist a frame $\taskframe$, model $\model$, history $\tau$, and time $t$ such that all formulas in $\context$ are true at $\model, \tau, t$.
\end{definition}

\begin{theorem}[Monotonicity]
Logical consequence is monotonic: if $\context \subseteq \Delta$ and $\context \models \varphi$, then $\Delta \models \varphi$.
\end{theorem}

\end{document}
