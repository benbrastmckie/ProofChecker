\documentclass[../BimodalReference.tex]{subfiles}
\begin{document}

\section{Task Semantics}

\subsection{Task Frames}

Task frames are the fundamental semantic structures for \textbf{TM}.
They abstract from universal laws governing transitions between world states while still retaining the temporal duration for a transition to complete.

The following primitives are required to define a task frame:

\begin{center}
\begin{tabular}{@{}lll@{}}
\toprule
Primitive & Type & Description \\
\midrule
$\worldstate$ & Type & World states \\
$D$ & Type & Temporal durations \\
$w \taskto{x} u$ & $\worldstate \to D \to \worldstate \to \text{Prop}$ & Task relation \\
\bottomrule
\end{tabular}
\end{center}

\begin{definition}[Task Frame]
A \textbf{task frame} over temporal type $D$ is a triple $\taskframe = (\worldstate, D, {\taskto{}})$ satisfying:
\begin{enumerate}
  \item \textbf{Nullity}: For all $w : \worldstate$, we have $w \taskto{0} w$.
  \item \textbf{Compositionality}: For all $w, u, v : \worldstate$ and $x, y : D$, if $w \taskto{x} u$ and $u \taskto{y} v$, then $w \taskto{x+y} v$.
\end{enumerate}
\end{definition}

\noindent
Nullity ensures that zero-duration tasks leave the world state unchanged.
Compositionality ensures that executing tasks sequentially yields results consistent with a single task of combined duration.

\subsection{World Histories}

A world history is a function from times to world states that respects the task relation over a convex temporal domain.
World histories represent possible paths through the space of world states.

\begin{definition}[Convex Domain]
A domain $\domain : D \to \text{Prop}$ is \textbf{convex} if whenever $a, c \in \domain$ with $a \le c$, every time $b$ with $a \le b \le c$ is also in $\domain$.
More precisely, for all $a, b, c : D$, if $\domain(a)$ and $\domain(c)$ and $a \le b \le c$, then $\domain(b)$.
Convexity ensures the domain has no temporal gaps.
\end{definition}

\begin{definition}[World History]
A \textbf{world history} in a task frame $\taskframe$ is a dependent function $\tau : (x : D) \to \domain(x) \to \worldstate$ where $\domain : D \to \text{Prop}$ is a convex subset of $D$ and $\tau(x) \taskto{y-x} \tau(y)$ for all times $x, y : D$ with $\domain(x)$, $\domain(y)$, and $x \le y$.
We write $\histories_{\taskframe}$ for all world histories over frame $\taskframe$.
\end{definition}

\subsection{Task Models}

A task model extends a task frame with an interpretation function that assigns truth values to sentence letters at world states.
\textbf{Propositions} are subsets of $\worldstate$ representing instantaneous ways for the system to be.
Sentence letters express propositions, which can be realized by zero or more world states.
World states themselves are specific configurations of the total system at an instant.

\begin{definition}[Task Model]
A \textbf{task model} defined over a frame $\taskframe$ is a pair $\model = (\taskframe, I)$ where the \textbf{interpretation function} $I : \worldstate \to \text{String} \to \text{Prop}$ assigns to each world state $w : \worldstate$ and sentence letter $p : \text{String}$ a truth value $I(w, p) : \text{Prop}$.
We write $I(w, p)$ to indicate that sentence letter $p$ is true at $w$.
\end{definition}

\subsection{Truth Conditions}

Truth is evaluated relative to a model $\model$ providing the interpretation, a world history $\tau$ representing a possible path through the space of world states, and a time $x : D$.
Whereas the model fixes the interpretation of the language, the contextual parameters $\tau$ and $x$ determine the truth value of every sentence of the langauge.

\begin{definition}[Truth]
For model $\model$, history $\tau \in \histories_{\taskframe}$, and time $x : D$:
\begin{align*}
  \model, \tau, x \vDash p &\Iff x \in \domain(\tau) \text{ and } I(\tau(x), p) \\
  \model, \tau, x \nvDash \falsum \\
  \model, \tau, x \vDash \varphi \imp \psi &\Iff
    \model, \tau, x \nvDash \varphi \text{ or } \model, \tau, x \vDash \psi \\
  \model, \tau, x \vDash \nec\varphi &\Iff
    \model, \sigma, x \vDash \varphi \text{ for all } \sigma \in \histories_{\taskframe} \\
  \model, \tau, x \vDash \allpast\varphi &\Iff
    \model, \tau, y \vDash \varphi \text{ for all } y : D \text{ where } y < x \\
  \model, \tau, x \vDash \allfuture\varphi &\Iff
    \model, \tau, y \vDash \varphi \text{ for all } y : D \text{ where } x < y
\end{align*}
\end{definition}

\noindent
The modal operator $\nec$ quantifies over all world histories $\sigma : \histories_{\taskframe}$ at the current time $x : D$.
The temporal operators $\allpast$ and $\allfuture$ quantify over all earlier and later times $y : D$.


\subsection{Time-Shift}

The time-shift operation is used to establish the \textbf{perpetuity principles}:
\begin{itemize}
  \item P1: $\nec\varphi \imp \always\varphi$ (what is necessary is always the case)
  \item P2: $\sometimes\varphi \imp \poss\varphi$ (what is sometimes the case is possible)
\end{itemize}
It is natural to assume that whatever is necessary is always the case, or equivalently, whatever is sometimes the case is possible.
Time-shift enables proofs of the bimodal axioms MF ($\nec\varphi \imp \nec\allfuture\varphi$) and TF ($\nec\varphi \imp \allfuture\nec\varphi$) which together imply the perpetuity principles.

\begin{definition}[Time-Shift]
For $\tau, \sigma \in \histories_{\taskframe}$ and $x, y : D$, world histories $\tau$ and $\sigma$ are \textbf{time-shifted from $y$ to $x$}, written $\tau \approx_y^x \sigma$, if and only if there exists an order automorphism $\bar{a} : D \to D$ where $y = \bar{a}(x)$, $\domain_\sigma = \bar{a}^{-1}(\domain_\tau)$, and $\sigma(z) = \tau(\bar{a}(z))$ for all $z \in \domain_\sigma$.
\end{definition}

\noindent
Time-shifting preserves the essential structure of histories:

\begin{theorem}[Convexity Preservation]
If $\tau$ has a convex domain and $\tau \approx_y^x \sigma$, then $\sigma$ has a convex domain.
\end{theorem}

\begin{theorem}[Task Preservation]
If $\tau$ respects the task relation and $\tau \approx_y^x \sigma$, then $\sigma$ respects the task relation.
\end{theorem}

\subsection{Logical Consequence and Validity}

Logical consequence and validity are defined uniformly across all temporal types, frames, models, histories, and times.

\begin{definition}[Logical Consequence]
A formula $\varphi$ is a \textbf{logical consequence} of $\Gamma$ (written $\Gamma \models \varphi$) just in case for every temporal type $D : \text{Type}$, frame $\taskframe : \text{TaskFrame}(D)$, model $\model : \text{TaskModel}(\taskframe)$, history $\tau \in \histories_{\taskframe}$, and time $x : D$, if $\model, \tau, x \models \psi$ for all $\psi \in \Gamma$, then $\model, \tau, x \models \varphi$.
\end{definition}

\begin{definition}[Validity]
A formula $\varphi$ is \textbf{valid} (written $\models \varphi$) just in case $\varphi$ is a logical consequence of the empty set: $\varnothing \models \varphi$.
Equivalently, $\varphi$ is true at every model-history-time triple.
\end{definition}

\begin{definition}[Satisfiability]
A context $\context$ is \textbf{satisfiable} in temporal type $D : \text{Type}$ if there exist a frame $\taskframe : \text{TaskFrame}(D)$, model $\model : \text{TaskModel}(\taskframe)$, history $\tau \in \histories_{\taskframe}$, and time $x : D$ such that $\model, \tau, x \models \psi$ for all $\psi \in \context$.
\end{definition}

\begin{theorem}[Monotonicity]
If $\context \subseteq \Delta$ and $\context \models \varphi$, then $\Delta \models \varphi$.
\end{theorem}

\end{document}
