\documentclass[../BimodalReference.tex]{subfiles}
\begin{document}

\section{Notes}

\subsection{Implementation Status}

\begin{center}
\begin{tabular}{@{}lll@{}}
\toprule
Component & Status & Notes \\
\midrule
Syntax & Complete & 6 primitives, derived operators \\
Semantics & Complete & Task frames, world histories, truth \\
Proof System & Complete & 14 axioms, 7 inference rules \\
Soundness & Fully Proven & All 14 axioms valid, 7 rules sound \\
Deduction Theorem & Fully Proven & Well-founded recursion on height \\
Completeness & Infrastructure Only & Axiomatized (Lindenbaum, truth lemma) \\
Decidability & Soundness Proven & Tableau-based, requires FMP for completeness \\
Perpetuity Principles & Fully Proven & P1-P6 all proven \\
\bottomrule
\end{tabular}
\end{center}

\subsection{Discrepancy Notes}

This section documents differences between the paper ``The Construction of Possible Worlds'' and the Lean implementation.

\subsubsection{Terminology}

\begin{itemize}
  \item The paper uses ``perpetuity principles'' for P1-P6; the Lean code uses the same terminology.
  \item The paper's notation $\triangle$ and $\triangledown$ for ``always'' and ``sometimes''
        is preserved in the Lean implementation as \texttt{always} and \texttt{sometimes}.
\end{itemize}

\subsubsection{Axiom Naming}

\begin{center}
\begin{tabular}{@{}lll@{}}
\toprule
Paper Name & Lean Name & Notes \\
\midrule
MT (Modal T) & \texttt{Axiom.modal\_t} & $\nec\varphi \imp \varphi$ \\
M4 (Modal 4) & \texttt{Axiom.modal\_4} & $\nec\varphi \imp \nec\nec\varphi$ \\
MB (Modal B) & \texttt{Axiom.modal\_b} & $\varphi \imp \nec\poss\varphi$ \\
MK & \texttt{Axiom.modal\_k\_dist} & K distribution \\
TK & \texttt{Axiom.temp\_k\_dist} & Temporal K distribution \\
T4 & \texttt{Axiom.temp\_4} & Temporal transitivity \\
TA & \texttt{Axiom.temp\_a} & Temporal connectedness \\
TL & \texttt{Axiom.temp\_l} & Temporal introspection \\
MF & \texttt{Axiom.modal\_future} & Modal-future interaction \\
TF & \texttt{Axiom.temp\_future} & Temporal-future interaction \\
\bottomrule
\end{tabular}
\end{center}

\subsubsection{M5 Collapse Axiom}

The implementation includes an explicit M5 collapse axiom (\texttt{Axiom.modal\_5\_collapse}):
\[
\poss\nec\varphi \imp \nec\varphi
\]
This is derivable from the other S5 axioms (MB + M4) but is included as a primitive
for proof convenience in the S5 collapse theorem.

\subsubsection{Temporal Type Generalization}

The paper uses a fixed temporal type $D = \mathbb{Z}$ (integers).
The implementation generalizes to any type $D$ with \texttt{LinearOrderedAddCommGroup} instance, allowing for integers, rationals, reals, or custom bounded time structures.

\subsubsection{Completeness Status}

The paper proves completeness via canonical model construction.
The Lean implementation has the completeness infrastructure (Lindenbaum lemma statement, canonical frame types, truth lemma statement) but the core lemmas remain axiomatized.
Estimated effort to complete: 70-90 hours of focused development.

\subsubsection{Decidability Implementation}

The implementation includes a tableau-based decision procedure for validity that provides an alternative to the canonical model approach.
The decidability module establishes that validity is decidable via constructive tableau expansion and branch closure.
Soundness is fully proven: if the procedure returns ``valid'', the formula is semantically valid.
Completeness is partial because it relies on the Finite Model Property (FMP), which is axiomatized.
The FMP states that if a formula is satisfiable, it is satisfiable in a finite model.
Full formalization of the FMP would complete the decidability proof.

\subsection{Source Files}

\begin{center}
\begin{tabular}{@{}ll@{}}
\toprule
Directory & Contents \\
\midrule
\texttt{Bimodal/Syntax/} & Formula, Context, temporal swap \\
\texttt{Bimodal/Semantics/} & TaskFrame, WorldHistory, Truth, Validity \\
\texttt{Bimodal/ProofSystem/} & Axioms, Derivation \\
\texttt{Bimodal/Metalogic/} & Soundness, DeductionTheorem, Completeness \\
\texttt{Bimodal/Metalogic/Decidability/} & SignedFormula, Tableau, Closure, Saturation \\
\texttt{Bimodal/Theorems/} & Perpetuity, ModalS5, Propositional, Combinators \\
\bottomrule
\end{tabular}
\end{center}

\end{document}
