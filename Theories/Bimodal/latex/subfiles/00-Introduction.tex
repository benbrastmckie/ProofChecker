\documentclass[../BimodalReference.tex]{subfiles}
\begin{document}

\section{Introduction}

Bimodal TM is a logic combining S5 metaphysical necessity ($\nec$) with linear
temporal operators for past ($\allpast$) and future ($\allfuture$). The logic
provides a framework for reasoning about necessary truths across time, with
applications to metaphysics, action theory, and formal philosophy.

The semantics is based on \emph{task frames}, which extend Kripke frames with
temporal structure. A task frame consists of world states connected by a
\emph{task relation} indexed by temporal durations. World histories are
temporal slices through a task frame, representing the unfolding of a world
over time.

\subsection*{Implementation Status}

\begin{itemize}
  \item \textbf{Syntax}: Complete (6 primitive constructors, derived operators)
  \item \textbf{Semantics}: Complete (task frames, world histories, truth conditions)
  \item \textbf{Proof System}: Complete (14 axiom schemata, 7 inference rules)
  \item \textbf{Soundness}: Fully proven (all axioms valid, rules preserve validity)
  \item \textbf{Completeness}: Infrastructure only (Lindenbaum lemma, canonical model axiomatized)
\end{itemize}

\subsection*{Source Code}

The Lean 4 implementation is in the \texttt{Bimodal/} directory:
\begin{itemize}
  \item \texttt{Syntax/Formula.lean} -- Formula type and operators
  \item \texttt{Semantics/} -- Task frames, world histories, truth conditions
  \item \texttt{ProofSystem/} -- Axioms and derivation trees
  \item \texttt{Metalogic/} -- Soundness and completeness
  \item \texttt{Theorems/} -- Perpetuity principles and modal theorems
\end{itemize}

\end{document}
