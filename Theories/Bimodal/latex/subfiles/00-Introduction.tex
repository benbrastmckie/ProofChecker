\documentclass[../BimodalReference.tex]{subfiles}
\begin{document}

\section{Introduction}

\textbf{TM} is a bimodal logic combining S5 a historical necessity operator ($\nec$) with linear temporal operators for past ($\allpast$) and future ($\allfuture$).
The logic provides a framework for reasoning about necessary truths across time to train AI systems to reason about past and future contingency.

The semantics is based on \emph{task frames}, which extend Kripke frames with temporal structure.
A task frame consists of world states connected by a \emph{task relation} indexed by temporal durations.
World histories are temporal slices through a task frame, representing the unfolding of a world over time.

\subsection*{Project Structure}

The Lean 4 implementation is in the \texttt{Bimodal/} directory:
\begin{itemize}
  \item \texttt{Syntax/} -- Defines the formula language with 6 primitive constructors and derived operators.
  \textbf{Complete.}
  \item \texttt{ProofSystem/} -- Axioms (14 schemata) and inference rules (7 rules) forming a Hilbert-style proof system.
  \textbf{Complete.}
  \item \texttt{Semantics/} -- Task frames model possible worlds; world histories model time; truth conditions define meaning.
  \textbf{Complete.}
  \item \texttt{Metalogic/} -- Soundness theorem (proven: all axioms valid, rules preserve validity), deduction theorem (proven: enables assumption introduction), and completeness infrastructure (Lindenbaum lemma, canonical model axiomatized).
  \textbf{Soundness and deduction complete; completeness pending.}
  \item \texttt{Theorems/} -- Perpetuity principles and modal theorems derived from the axiom system.
  \textbf{Partial.}
\end{itemize}

\end{document}
