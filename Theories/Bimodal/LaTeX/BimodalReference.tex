% ============================================================================
% BimodalReference.tex
% Bimodal TM Logic: A Reference Manual
%
% This document provides the formal specification of the Bimodal TM logic,
% a bimodal logic combining S5 metaphysical modality with linear temporal
% operators, as implemented in the Bimodal/ directory.
% ============================================================================

\documentclass[11pt]{article}

% ============================================================================
% Packages
% ============================================================================

% Mathematics
\usepackage{amsmath}
\usepackage{amsthm}
\usepackage{amssymb}

% Document structure
\usepackage{subfiles}

% Tables
\usepackage{booktabs}
\usepackage{array}

% Lists
\usepackage{enumitem}

% Custom packages (found via TEXINPUTS configured in latexmkrc)
\usepackage{bimodal-notation}
\usepackage{formatting}

% References
\usepackage{cleveref}

% ============================================================================
% Theorem Environments
% ============================================================================

\theoremstyle{definition}
\newtheorem{definition}{Definition}[section]

\theoremstyle{plain}
\newtheorem{theorem}[definition]{Theorem}
\newtheorem{lemma}[definition]{Lemma}
\newtheorem{axiom}[definition]{Axiom}

\theoremstyle{remark}
\newtheorem{remark}[definition]{Remark}

% ============================================================================
% Document Info
% ============================================================================

% Custom horizontal rule command
\newcommand{\HRule}{\rule{\linewidth}{0.5mm}}

% ============================================================================
% Begin Document
% ============================================================================

\begin{document}

% Custom title page
\begin{titlepage}
\begin{center}

\vspace*{2cm}

\HRule\\[0.4cm]
{\Huge \bfseries Bimodal Reference Manual}\\[0.2cm]
\HRule\\[1cm]

{\LARGE\itshape A Logic for Tense and Modality}\\[1cm]


{\large\itshape Benjamin Brast-McKie}\\[0.15cm]
\texttt{\href{https://www.benbrastmckie.com}{www.benbrastmckie.com}}\\[0.15cm]
{--- \today\ ---}\\[1cm]

\vfill

{\normalsize\bfseries Primary Reference:}\\[0.3cm]
  \href{https://www.benbrastmckie.com/wp-content/uploads/2025/11/possible_worlds.pdf}{\textit{``The Construction of Possible Worlds''}}, Brast-McKie, (under review), 2025.\\[1cm]

\end{center}
\end{titlepage}
\thispagestyle{empty}

{

\begin{abstract}
\noindent
  This reference manual provides the formal specification of the Bimodal logic TM for tense and modality as implemented in the \texttt{ProofChecker} project.
  TM is a bimodal logic combining S5 metaphysical necessity with linear temporal operators for past and future.
  The soundness theorem is fully proven; completeness infrastructure is in place but the core lemmas remain axiomatized.
\end{abstract}

\pagestyle{empty}
\tableofcontents
\cleardoublepage
}

% ============================================================================
% Content
% ============================================================================

\subfile{subfiles/00-Introduction}
\subfile{subfiles/01-Syntax}
\subfile{subfiles/02-Semantics}
\subfile{subfiles/03-ProofTheory}
\subfile{subfiles/04-Metalogic}
\subfile{subfiles/05-Theorems}
\subfile{subfiles/06-Notes}

\end{document}
