\documentclass[../BimodalReference.tex]{subfiles}
\begin{document}

\section{Proof Theory}

\subsection{Axiom Schemata}

The TM proof system has 14 axiom schemata.

\subsubsection{Propositional Axioms}

\begin{axiom}[K (Distribution)]
$((\varphi \imp (\psi \imp \chi)) \imp ((\varphi \imp \psi) \imp (\varphi \imp \chi)))$
\end{axiom}

\begin{axiom}[S (Weakening)]
$(\varphi \imp (\psi \imp \varphi))$
\end{axiom}

\begin{axiom}[EFQ (Ex Falso)]
$(\falsum \imp \varphi)$
\end{axiom}

\begin{axiom}[Peirce]
$(((\varphi \imp \psi) \imp \varphi) \imp \varphi)$
\end{axiom}

\subsubsection{Modal Axioms (S5)}

\begin{axiom}[MT (Modal T)]
$(\nec\varphi \imp \varphi)$
\end{axiom}

\begin{axiom}[M4 (Modal 4)]
$(\nec\varphi \imp \nec\nec\varphi)$
\end{axiom}

\begin{axiom}[MB (Modal B)]
$(\varphi \imp \nec\poss\varphi)$
\end{axiom}

\begin{axiom}[M5 (Modal 5 Collapse)]
$(\poss\nec\varphi \imp \nec\varphi)$
\end{axiom}

\begin{axiom}[MK (Modal K Distribution)]
$(\nec(\varphi \imp \psi) \imp (\nec\varphi \imp \nec\psi))$
\end{axiom}

\subsubsection{Temporal Axioms}

\begin{axiom}[TK (Temporal K Distribution)]
$(\allfuture(\varphi \imp \psi) \imp (\allfuture\varphi \imp \allfuture\psi))$
\end{axiom}

\begin{axiom}[T4 (Temporal 4)]
$(\allfuture\varphi \imp \allfuture\allfuture\varphi)$
\end{axiom}

\begin{axiom}[TA (Temporal A)]
$(\varphi \imp \allfuture\somepast\varphi)$
\end{axiom}

\begin{axiom}[TL (Temporal L)]
$(\always\varphi \imp \allfuture\allpast\varphi)$
\end{axiom}

\subsubsection{Modal-Temporal Interaction}

\begin{axiom}[MF (Modal-Future)]
$(\nec\varphi \imp \nec\allfuture\varphi)$
\end{axiom}

\begin{axiom}[TF (Temporal-Future)]
$(\nec\varphi \imp \allfuture\nec\varphi)$
\end{axiom}

\begin{center}
\begin{tabular}{@{}lll@{}}
\toprule
Axiom & Lean Constructor & Pattern \\
\midrule
K & \texttt{Axiom.prop\_k} & Distribution \\
S & \texttt{Axiom.prop\_s} & Weakening \\
EFQ & \texttt{Axiom.ex\_falso} & Explosion \\
Peirce & \texttt{Axiom.peirce} & Classical \\
MT & \texttt{Axiom.modal\_t} & Reflexivity \\
M4 & \texttt{Axiom.modal\_4} & Transitivity \\
MB & \texttt{Axiom.modal\_b} & Symmetry \\
M5 & \texttt{Axiom.modal\_5\_collapse} & S5 collapse \\
MK & \texttt{Axiom.modal\_k\_dist} & Modal distribution \\
TK & \texttt{Axiom.temp\_k\_dist} & Temporal distribution \\
T4 & \texttt{Axiom.temp\_4} & Temporal transitivity \\
TA & \texttt{Axiom.temp\_a} & Connectedness \\
TL & \texttt{Axiom.temp\_l} & Introspection \\
MF & \texttt{Axiom.modal\_future} & Modal-future \\
TF & \texttt{Axiom.temp\_future} & Temporal-modal \\
\bottomrule
\end{tabular}
\end{center}

\subsection{Inference Rules}

The proof system has 7 inference rules.

\begin{definition}[Axiom Rule]
If $\varphi$ matches an axiom schema, then $\context \proves \varphi$.
\end{definition}

\begin{definition}[Assumption Rule]
If $\varphi \in \context$, then $\context \proves \varphi$.
\end{definition}

\begin{definition}[Modus Ponens]
\[
\frac{\context \proves \varphi \imp \psi \qquad \context \proves \varphi}{\context \proves \psi}
\]
\end{definition}

\begin{definition}[Necessitation]
\[
\frac{\proves \varphi}{\proves \nec\varphi}
\]
Applies only to theorems (empty context).
\end{definition}

\begin{definition}[Temporal Necessitation]
\[
\frac{\proves \varphi}{\proves \allfuture\varphi}
\]
Applies only to theorems (empty context).
\end{definition}

\begin{definition}[Temporal Duality]
\[
\frac{\proves \varphi}{\proves \text{swap}(\varphi)}
\]
Applies only to theorems (empty context).
\end{definition}

\begin{definition}[Weakening]
\[
\frac{\context \proves \varphi \qquad \context \subseteq \Delta}{\Delta \proves \varphi}
\]
\end{definition}

\begin{center}
\begin{tabular}{@{}lll@{}}
\toprule
Rule & Lean Constructor & Context Requirement \\
\midrule
Axiom & \texttt{DerivationTree.axiom} & Any \\
Assumption & \texttt{DerivationTree.assumption} & Any \\
Modus Ponens & \texttt{DerivationTree.modus\_ponens} & Any \\
Necessitation & \texttt{DerivationTree.necessitation} & Empty only \\
Temp. Necessitation & \texttt{DerivationTree.temporal\_necessitation} & Empty only \\
Temporal Duality & \texttt{DerivationTree.temporal\_duality} & Empty only \\
Weakening & \texttt{DerivationTree.weakening} & Any \\
\bottomrule
\end{tabular}
\end{center}

\subsection{Derivation Trees}

Derivations are represented as inductive trees.

\begin{definition}[Derivation Tree]
\texttt{DerivationTree $\Gamma$ $\varphi$} (written $\context \proves \varphi$) is an inductive type
representing a derivation of $\varphi$ from context $\context$.
\end{definition}

\begin{definition}[Height]
The height of a derivation tree:
\begin{itemize}
  \item Base cases (axiom, assumption): height 0
  \item Unary rules: height of subderivation + 1
  \item Modus ponens: max of both subderivations + 1
\end{itemize}
\end{definition}

The height measure enables well-founded recursion in metalogical proofs.

\subsection{Notation}

\begin{center}
\begin{tabular}{@{}ll@{}}
\toprule
Notation & Lean \\
\midrule
$\context \proves \varphi$ & \texttt{DerivationTree $\Gamma$ $\varphi$} \\
$\proves \varphi$ & \texttt{DerivationTree [] $\varphi$} \\
\bottomrule
\end{tabular}
\end{center}

\end{document}
