% ============================================================================
% BimodalReference.tex
% Bimodal TM Logic: A Reference Manual
%
% This document provides the formal specification of the Bimodal TM logic,
% a bimodal logic combining S5 metaphysical modality with linear temporal
% operators, as implemented in the Bimodal/ directory.
% ============================================================================

\documentclass[11pt]{article}

% ============================================================================
% Packages
% ============================================================================

% Mathematics
\usepackage{amsmath}
\usepackage{amsthm}
\usepackage{amssymb}

% Document structure
\usepackage{subfiles}

% Tables
\usepackage{booktabs}
\usepackage{array}

% Lists
\usepackage{enumitem}

% Custom packages
\usepackage{assets/bimodal-notation}
\usepackage{assets/formatting}

% References
\usepackage{cleveref}

% ============================================================================
% Theorem Environments
% ============================================================================

\theoremstyle{definition}
\newtheorem{definition}{Definition}[section]

\theoremstyle{plain}
\newtheorem{theorem}[definition]{Theorem}
\newtheorem{lemma}[definition]{Lemma}
\newtheorem{axiom}[definition]{Axiom}

\theoremstyle{remark}
\newtheorem{remark}[definition]{Remark}

% ============================================================================
% Document Info
% ============================================================================

\title{Bimodal TM Logic: A Reference Manual}
\author{ProofChecker Project}
\date{\today}

% ============================================================================
% Begin Document
% ============================================================================

\begin{document}

\maketitle
\thispagestyle{empty}

\begin{abstract}
\noindent
This reference manual provides the formal specification of the Bimodal TM logic
as implemented in the ProofChecker project. TM is a bimodal logic combining
S5 metaphysical necessity with linear temporal operators for past and future.
The soundness theorem is fully proven; completeness infrastructure is in place
but the core lemmas remain axiomatized.
\end{abstract}

{
\pagestyle{empty}
\tableofcontents
\cleardoublepage
}

% ============================================================================
% Content
% ============================================================================

\subfile{subfiles/00-Introduction}
\subfile{subfiles/01-Syntax}
\subfile{subfiles/02-Semantics}
\subfile{subfiles/03-ProofTheory}
\subfile{subfiles/04-Metalogic}
\subfile{subfiles/05-Theorems}
\subfile{subfiles/06-Notes}

\end{document}
