% ============================================================================
% LogosReference.tex
% Logos: A Reference Manual
%
% This document provides the formal specification of the Logos logic system,
% mirroring the semantic specification in RECURSIVE_SEMANTICS.md and the
% Lean 4 implementation in the Logos/ directory.
% ============================================================================

\documentclass[11pt]{article}

% ============================================================================
% Packages
% ============================================================================

% Mathematics
\usepackage{amsmath}
\usepackage{amsthm}
\usepackage{amssymb}
\usepackage{stmaryrd}

% Document structure
\usepackage{subfiles}

% Tables
\usepackage{booktabs}
\usepackage{array}

% Lists
\usepackage{enumitem}

% Graphics (for diagrams)
\usepackage{tikz}

% Custom packages
\usepackage{assets/logos-notation}
\usepackage{../../LaTeX/formatting}

% References (cleveref must be loaded after hyperref, which formatting.sty loads)
\usepackage{cleveref}

% ============================================================================
% Theorem Environments
% ============================================================================

\theoremstyle{definition}
\newtheorem{definition}{Definition}[section]
\newtheorem{example}[definition]{Example}

\theoremstyle{plain}
\newtheorem{theorem}[definition]{Theorem}
\newtheorem{lemma}[definition]{Lemma}
\newtheorem{proposition}[definition]{Proposition}
\newtheorem{corollary}[definition]{Corollary}

\theoremstyle{remark}
\newtheorem{remark}[definition]{Remark}
\newtheorem{notation}[definition]{Notation}

% Custom environment for open questions
\newenvironment{question}
  {\begin{quote}\textsc{[Open Question]:}}
  {\end{quote}}

% ============================================================================
% Document Info
% ============================================================================

% Custom horizontal rule command
\newcommand{\HRule}{\rule{\linewidth}{0.5mm}}

% ============================================================================
% Begin Document
% ============================================================================

\begin{document}

% Custom title page
\begin{titlepage}
\begin{center}

\vspace*{2cm}

\HRule\\[0.4cm]
{\Huge \bfseries Logos Reference Manual}\\[0.2cm]
\HRule\\[1cm]

{\LARGE\itshape A Logic for Verified and Interpreted AI Reasoning}\\[1cm]


{\large\itshape Benjamin Brast-McKie}\\[0.15cm]
\texttt{\href{https://www.benbrastmckie.com}{www.benbrastmckie.com}}\\[0.15cm]
{--- \today\ ---}\\[1cm]

\vfill

{\normalsize Primary References:}\\[0.3cm]
Brast-McKie, \href{https://link.springer.com/article/10.1007/s10992-025-09793-8}{\textit{Counterfactual Worlds}}, \textit{J.~Phil.~Logic}, 2025.\\[0.2cm]
Brast-McKie, \href{https://link.springer.com/article/10.1007/s10992-021-09612-w}{\textit{Identity and Aboutness}}, \textit{J.~Phil.~Logic}, 2021.\\[1cm]

\end{center}
\end{titlepage}
\thispagestyle{empty}

{

\begin{abstract}
\noindent
This reference manual provides the formal specification of the Logos logic system.
Logos is a hyperintensional modal-temporal logic with exact truthmaker semantics,
designed for reasoning about necessity, possibility, time, and counterfactual
conditionals. The semantics extends from a mereological state space foundation
through increasingly expressive extensions.
\end{abstract}

\pagestyle{empty}
\tableofcontents
\cleardoublepage
}

% ============================================================================
% Foundation Layer
% ============================================================================

\subfile{subfiles/00-Introduction}
\subfile{subfiles/01-ConstitutiveFoundation}

% ============================================================================
% Explanatory Extension
% ============================================================================

\subfile{subfiles/02-ExplanatoryExtension-Syntax}
\subfile{subfiles/03-ExplanatoryExtension-Semantics}
\subfile{subfiles/04-ExplanatoryExtension-Axioms}

% ============================================================================
% Future Extensions (stub content)
% ============================================================================

\subfile{subfiles/05-Epistemic}
\subfile{subfiles/06-Normative}
\subfile{subfiles/07-Spatial}
\subfile{subfiles/08-Agential}

% ============================================================================
% Back Matter
% ============================================================================

\nocite{*}  % Include all bibliography entries without explicit citations
\bibliographystyle{bib_style}
\bibliography{bibliography/LogosReferences}

\end{document}
